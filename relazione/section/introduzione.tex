\section{Descrizione del problema}
Il progetto consiste nel realizzare un software che simuli una macchina a controllo numerico (CNC - Computer Numerical Control) per la fresatura di blocchi di materiale a forma di parallelepipedo rettangolo.

Le specifiche date richiedono che il progetto sia eseguibile sia in ambiente Microsoft Windows (Visual Studio) che Linux ed è stato fornito un diagramma UML con le principali classi da implementare. Il simulatore dovrà accettare in ingresso un file di testo contenente la configurazione degli agenti -ovvero le specifiche della punta della fresa e del blocco- e una lista di \emph{posizioni}: decine di valori che rappresentano la roto-traslazione del blocco di materiale e della fresa nello spazio. Questo file, assieme ad un valore numerico che esprime la precisione della lavorazione, costituisce l'input per il software che dovrà essere in grado di elaborare i movimenti richiesti, asportare le porzioni di blocco corrette, e mostrare a video l'avanzamento della fresatura.
