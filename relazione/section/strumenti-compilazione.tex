\section{Strumenti usati, prerequisiti e istruzioni}
\subsection{Strumenti usati}
Il progetto è stato sviluppato in C++. Per lo sviluppo in ambiente Linux abbiamo usato Eclipse su Ubuntu 12.04, mentre per l'ambiente Windows è stato usato Visual Studio 2010. Lo strumento usato per la compilazione è CMake ($\geq 2.6$).

\subsection{Prerequisiti}
Il progetto è stato sviluppato usando le seguenti librerie:
\begin{itemize}[noitemsep]
  \item Boost ($\geq 1.48$)
  \item Eigen ($\geq 3.1.1$)
  \item OpenSceneGraph ($\geq 3.0.0$)
\end{itemize}

\subsection{Istruzioni}
Per compilare il progetto bisogna seguire i seguenti passi:
\begin{enumerate}
  \item portarsi nella cartella \verb!/path/del/progetto/!
  \item lanciare il comando \verb!cmake source/CMakeLists.txt!
    \begin{itemize}[noitemsep]
      \item se in Windows, aggiungere il flag \verb!-G"Visual Studio 10"!
      \item se in linux, aggiugnere il flag \verb!-G"Unix Makefiles"!
    \end{itemize}
    \begin{itemize}[noitemsep]
      \item se si vuole compilare in modalità \verb!debug!, aggiungere il flag \verb!-D CMAKE_BUILD_TYPE=Debug!
      \item se si vuole compilare in modalità \verb!release!, aggiungere il flag \verb!-D CMAKE_BUILD_TYPE=Release!
    \end{itemize}
  \item lanciare il comando \verb!make! per compilare.
\end{enumerate}

Per eseguire il progetto, lanciare il comando\\ \verb!/path/del/progetto/CNCSimulator opzioni file_positions!\\ dove:
\begin{itemize}
  \item le opzioni possono essere:
    \begin{enumerate}[noitemsep]
      \item \verb!-s x!, dove \verb!x! è la dimensione minima dei voxel. Minore è \verb!x!, maggiore è la precisione della simulazione
      \item \verb!-v box|mesh|none!, per specificare il tipo di visualizzazione
      \item \verb!-p! lancia il simulatore in pausa
      \item \verb!-f x!, dove \verb!x! indica il rate di apertura del getto d'acqua per la rimozione dei detriti in eccesso
      \item \verb!-t x!, dove \verb!x! indica la quantità di materiale da rimuovere prima di attivare il getto d'acqua
      \item \verb!-h!, per visualizzare il menu di help completo
    \end{enumerate}
  \item \verb!file_positions! è il file contenente i movimenti da riprodurre.
\end{itemize}

\section{Esempio di lavorazione}

