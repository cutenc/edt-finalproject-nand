\subsection{Mesher}
\label{sec:modules_mesher}

Il mesher è il componente che, a partire dalla lavorazione effettuata dal miller, crea la mesh 3D dell'oggetto lavorato, traducendolo quindi in un oggetto tridimensionale da visualizzare.

Quando è pronto a eseguire del lavoro, richiede all’oggetto che rappresenta il prodotto le ultime modifiche effettuate: per tutte le foglie cancellate o aggiornate vengono eseguite le relative operazioni in maniera diretta, sfruttando il puntatore all’oggetto grafico contenuto;  quelle nuove, invece, vengono inserite nella scena solo se effettivamente visibili. Per facilitare il processo di visualizzazione, la parte di scena rappresentante il prodotto viene modellata con un octree le cui foglie contengono una molteplicità di voxel da rappresentare: questa scelta permette di ottimizzare l’uso delle risorse grafiche in quanto si diminuisce il numero di mesh, aumentandone la dimensione.

Il mesher esegue l'algoritmo Marching Cubes per creare la mesh a partire dai voxel. Marching Cubes permette inoltre di scartare dalla mesh quei voxel che sicuramente non sono visibili, perché totalmente interni o totalmente esterni (quindi eliminati) alla mesh.


