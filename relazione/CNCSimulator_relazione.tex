\documentclass[a4paper,12pt]{article}

\usepackage{enumitem}
\usepackage[utf8]{inputenc}
\usepackage{textcomp}
\usepackage{xspace}
\usepackage[english,italian]{babel}
\usepackage[pdftex]{graphicx}
\usepackage{fullpage}
\usepackage{amsmath}
\usepackage{caption}
\usepackage{subcaption}
\usepackage{multirow}
\usepackage{amsfonts}

\usepackage{listings}
\lstset{
	basicstyle=\fontsize{8}{12}\ttfamily,
	inputencoding=utf8,
	language=C++,
	numbers=left,
	numberstyle=\tiny,
	tabsize=2,
	frame=single,
	backgroundcolor=\color{gray},
}

\setenumerate[2]{label=\alph*.}

\usepackage{color}
\definecolor{gray}{gray}{0.9}

\usepackage{hyperref}
\hypersetup{
	colorlinks=true
}
\usepackage{hypcap}

% qualche comando utile che evita di dover scrivere parole per intero



% qualche comando per uniformare lo stile di scrittura
\newcommand{\shell}[1]{\texttt{#1}}

\begin{document}

\thispagestyle{empty}
\begin{center}
	\leavevmode
	\large
	\begin{tabular}{ r l }
		\multirow{2}{*}{\includegraphics[width=2cm]{img/unipd_logo.png}} & \textsc{Università degli studi di Padova}\par \\
			& \textsc{Corso di laurea magistrale in Ingegneria Informatica} \\
	\end{tabular}
		
	\vfill
	{\Large Elaborazione di Dati Tridimensionali \par Relazione del progetto finale} \par
	\vskip .5cm
	\textbf{{\LARGE CNC Simulator}}\par
	\vskip 3cm
	\normalfont
	
	\begin{tabular}{ c c }
		\large Alberto \textsc{Franzin} & \large Nicola \textsc{Gobbo} \\
		\normalsize 1012883 & \normalsize 1014195 \\[1cm]
	\end{tabular}
	\normalfont
	\vskip 4cm
	
	\begin{flushright}
		\emph{Docente:}\\
		Prof. Emanuele \textsc{Menegatti}\\
	\end{flushright}

	
	\vfill
	{\large AA 2011-12}
\end{center}
\cleardoublepage


\begingroup
	\hypersetup{linkcolor=black}
	\setcounter{tocdepth}{2}
	\tableofcontents
\endgroup

\newpage

\section{Descrizione del problema}
Il progetto consiste nello scrivere un simulatore di macchina a controllo numerico (CNC - Computer Numerical Control) per il controllo della fresatura di un blocco di materiale.

Le specifiche fornite richiedono che il progetto sia eseguibile sia in ambiente Windows/Visual Studio che Linux. Viene anche fornito un diagramma con le principali classi da implementare, per uniformità. Viene richiesto che il simulatore accetti in ingresso un file contenente una lista di posizioni del blocco di materiale e dell'utensile di lavorazione, e la precisione della lavorazione, e che sia in grado di elaborare i movimenti comportati dalla lavorazione e di mostrare l'avanzamento della fresatura.



\section{Descrizione dei moduli}
Nel nostro svolgimento, abbiamo suddiviso il lavoro nei seguenti moduli, che andiamo a descrivere.


\subsection{UML del moduli}
{\Large inserire qui l'immagine UML dei moduli}

\subsection{Configurator}
\code{configurator} è il modulo che si occupa di leggere i dati di ingresso e trasformarli in strutture dati comprensibili al resto del programma. Le fonti da cui attinge le informazioni sono la linea di comando, attraverso la classe \code{CommandLineParser}, e il file di configurazione, attraverso la classe \code{ConfigFileParser}.

\code{CommandLineParser} deve tutta la sua flessibilità nell'acquisizione della linea di comando alla libreria \boost \code{program_options} di cui la classe è un semplice wrapper. Altro discorso va fatto per \code{ConfigFileParser}, classe scritta ad-hoc, in quanto il file da interpretare era di tipo \emph{plain-text} non strutturato. Per garantire una certa flessibilità al contenuto del file, questa classe permette di:
\begin{itemize}
	\item invertire la posizione delle sezioni \code{[PRODUCT]} e \code{[TOOL]}, fermo restando che la sezione \code{[POINTS]} deve rimanere l'ultima del file;
	\item gestire correttamente linee vuote o di commento, ovvero righe in cui il primo carattere non di spaziatura è ``\code{#}'';
	\item gestire parametri opzionali come la presenza o meno della direttiva \code{COLOR} nella sezione \code{[TOOL]}.
\end{itemize}
Individuata la sezione \code{[POINTS]}, \code{ConfigFileParser} demanda il compito di interpretare la lista delle posizioni a \code{CNCMoveIterator}. Questa classe estende \code{std::istream_iterator}, che a sua volta incarna il pattern \code{InputIterator}, proprio del \cpp: può perciò essere usata come un iteratore che, ad ogni dereferenziazione, legge la riga successiva del file e la interpreta come una coppia di roto-traslazioni, ritornando al chiamante queste informazioni con oggetti opportuni.

\paragraph{Sviluppi futuri.}
L'implementazione della classe \code{ConfigFileParser} utilizza solo funzioni definite nello standard \cpp{}03 ma, nonostante questo, esistono delle incongruenze nella gestione dei file testuali da parte di Windows e Linux, dovute in primo luogo al diverso marcatore di fine riga dei due sistemi operativi. Queste incongruenze interessano per lo più la gestione dei parametri opzionali e, in ambiente Windows, portano ad una errata interpretazione del file di configurazione. Per evitare problemi di questo tipo ed aumentare la flessibilità della configurazione, si consiglia di scomporre il file attuale in due parti: un primo file contenente la configurazione della fresatura, codificata in formato XML e un secondo file contenente la lista dei ``punti'' che, dovendo essere letta in modo sequenziale, può mantenere l'attuale formato.


\subsection{Miller}
Il \emph{miller} è il componente che, sulla linea di quanto descritto da Yau \textit{et al.} \cite{yau_CNCSimulation}, simula la fresatura vera e propria verificando dove e come l'utensile della macchina compenetra il blocco di materiale, determinando la porzione da rimuovere e decidendo se sia o meno necessario attivare il getto d'acqua di pulitura.

Il prodotto da lavorare (classe \code{Stock}) e l'utensile (classe \code{Cutter}) sono i principali attori di questo modulo. Per gestire in maniera efficiente il processo di erosione lo stock è stato modellato con una particolare struttura dati chiamata \emph{octree}. Un octree è un albero di arietà 8 ---in questo caso, non bilanciato--- che, come si evince dalla figura \ref{fig:octree_explanation}, permette di segmentare uno spazio tridimensionale in regioni via via più piccole man mano che la aumenta la profondità. Ogni foglia rappresenta quindi un parallelepipedo di volume, detto \emph{voxel}, e su di essa è salvato un valore che identifica quali vertici sono stati asportati dal cutter e quali, invece, sono ancora presenti. Per motivi di performance a ciò si aggiungono le informazioni necessarie a calcolare le coordinate dei vertici stessi ed un collegamento alle strutture adibite alla visualizzazione grafica del blocco, come spiegato nella sezione \ref{sec:modules_mesher}.
\begin{figure}[htp]
	\centering
	\includegraphics[width=.75\textwidth]{img/octree_explanation}
	\caption{Suddivisione dello spazio tridimensionale tramite albero octree. {\footnotesize (fonte immagine: \url{http://www.forceflow.be/wp-content/uploads/2012/04/octree1.gif})}}
	\label{fig:octree_explanation}
\end{figure}

All'interno del modulo di milling il cutter è caratterizzato da due soli parametri: una \emph{funzione di distanza} e una \emph{bounding box}. La funzione di distanza, descritta nell'equazione \eqref{equ:distance_function} indica se un punto dello spazio è interno o esterno all'utensile: nel caso sia interno, significa che è stato asportato.
\begin{equation} \label{equ:distance_function}
	distance(P) = 
		\begin{cases}
			< 0 & \text{se $P$ è esterno al cutter} \\
			\geq 0 & \text{se $P$ è interno al cutter}
		\end{cases}
	\qquad \text{ dove } P \in \mathbb{R}^3
\end{equation}
L'ingombro del cutter viene modellato attraverso un parallelepipedo orientato. La scelta di questa forma è dovuta al fatto che anche i voxel che costituiscono lo stock sono dei parallelepipedi: il problema dell'intersezione tra box è molto studiato in letteratura \cite{eberly_OBBCollision}, e sono stati individuati algoritmi efficienti per questo tipo di verifica. L'orientamento, invece, serve a poter definire le più piccole dimensioni possibili tali da contenere il cutter: ciò permette di eseguire dei test di intersezione meno rapidi ma anche più accurati.

\subsubsection{Il processo di erosione.}
Per ogni ``mossa'' letta da file, il \emph{miller} converte le due rototraslazioni in una isometria tridimensionale del cutter nei confronti del sistema di riferimento del prodotto. L'algoritmo attraversa quindi l'octree per individuare tutti e soli i voxel che intersecano il volume di riferimento dello stesso cutter: i rami da percorrere sono scelti in base a diverse funzioni di intersezione che diventano via via meno precise, ma più veloci, via via che aumenta la profondità e, di conseguenza, il numero di voxel da analizzare. Giunto ad una foglia dell'albero, l'algoritmo ne enumera i vertici e, per ognuno di essi, invoca la funzione $distance$ del cutter: così facendo si marcano i punti interni alla superficie di taglio che, da quel momento in avanti, verranno considerati erosi.
Le foglie rimaste prive di vertici vengono quindi eliminate, mentre per le altre, se la profondità massima non è ancora stata raggiunta, l'algoritmo effettua una divisione in otto parti del volume di competenza, aggiungendo un nuovo livello all'albero. Come scelta progettuale si è deciso di non condividere i vertici comuni tra voxel contigui in quanto il concetto di vicinanza spaziale non viene modellato bene dalla struttura octree, soprattutto se sbilanciata. Il costo computazionale necessario a recuperare i voxel ``vicini'', infatti, sarebbe stato superiore ai vantaggi portati dalla condivisione dei vertici stessi. Al termine di ogni mossa, il \emph{miller} conteggia la quantità di materiale eroso e non ancora pulito e decide se attivare o meno il getto d'acqua: la scelta viene presa tramite una funzione a doppia soglia, caratterizzata da un ``rate di pulitura'', cioè dal volume di detriti che l'acqua riesce a pulire per ogni mossa.

Mostrare a video lo stato dell'erosione comporta uno scambio di informazioni tra \emph{miller} e \emph{mesher} in quanto questi due algoritmi operano in modo indipendente e con diversi tempi di elaborazione. Per gestire in maniera efficiente l'accesso concorrente ai dati, ogniqualvolta una foglia viene cancellata il \emph{miller} la inserisce in una lista opportuna mentre, per le foglie aggiunte o modificate, il percorso da esse alla radice viene evidenziato. Così facendo il processo di \emph{meshing}, dopo aver acquisito il controllo esclusivo dell'octree, potrà ricavare rapidamente tutte e sole le foglie modificate dalla sua ultima visita. Per impedire che il processo di \emph{milling} possa subire starvation dal \emph{mesher}, quest'ultimo viene attivato al più al termine di ogni mossa letta da file e, comunque, non più di 30 volte al secondo: nei casi reali il tempo di attesa forzata del \emph{miller} è ridotto in quanto, un ciclo di rendering impiega molto più tempo della simulazione di una singola posizione e quest'ultima è più lenta dell'attraversamento dell'albero sui percorsi evidenziati\footnote{I rapporti tra le durate delle operazioni indicate variano di molto in base alla profondità massima dell'albero e alla ``mobilità'' dell'utensile, ovvero al numero di voxel modificati ad ogni iterazione.}.

\subsubsection{Miglioramenti tentati e sviluppi futuri.}
Il processo di \emph{milling} è frutto di più riscritture successive, ognuna delle quali ha sperimentato un modo diverso per rendere più efficiente e veloce l'algoritmo. Il primo approccio scorreva l'albero per livelli sfruttando una coda come struttura dati di appoggio in cui inserire i nodi da analizzare. L'idea di fondo era facilitare la successiva parallelizzazione dell'algoritmo in cui vari thread avrebbero usato la coda per gestire i nodi da controllare, così come mostrato in figura \ref{fig:queue_parallel}.
\begin{figure}[htp]
	\centering
	\includegraphics[width=.5\textwidth]{img/queue_parallel}
	\caption{Schematizzazione della prima idea di algoritmo parallelo che risolve il milling.}
	\label{fig:queue_parallel}
\end{figure}

Affrontare una simile strada avrebbe comportato un uso meno efficiente della memoria: l'inserimento dei nodi in una struttura dati in rapida espansione avrebbe comportato molti più accessi in RAM rispetto ad una semplice visita \emph{depth-first} in quanto la gerarchia di cache degli attuali elaboratori non sarebbe stata sfruttata appieno. La speranza era che il parallelismo riuscisse a sopperire a questo problema strutturale e, anzi, ad essere più rapido nell'esecuzione. Questo purtroppo non è avvenuto e, dai confronti con le implementazioni di altri colleghi, le prestazioni risultavano molto inferiori. Le ipotesi sulle motivazioni di questo ``fallimento'' sono state molteplici, ma la più plausibile risiede nell'aver scelto un approccio errato alla parallelizzazione, che male si adatta alla particolare struttura del problema: i nodi che devono venir analizzati sono molti ma, il lavoro da compiere su ognuno di essi è poco, per cui l'impianto necessario a gestire i thread introduce un overhead troppo elevato.

L'esperienza così maturata ha permesso di individuare un approccio migliore nella parallelizzazione del problema, analizzando l'octree attraverso il pattern Fork-Join\footnote{\url{http://www.oracle.com/technetwork/articles/java/fork-join-422606.html}} congiuntamente a uno thread-pool che permetta il \emph{work-stealing}. Così facendo si conservano tutti i vantaggi della visita \emph{depth-first}, senza relegare i thread all'analisi di singoli nodi, e beneficiando sia della possibilità di scrivere codice ricorsivo che di un processing embarassingly parallel.

Purtroppo per carenze di tempo e per la mancanza di librerie che implementano questo paradigma, si è optato per la classica versione ricorsiva. Questa risulta essere la più performante in single-thread ma, dal punto di vista dello spazio occupato, l'implementazione potrebbe essere ulteriormente migliorata, sfruttando i vantaggi offerti dalla ricorsione: attualmente, per esempio, ogni nodo dell'albero contiene un riferimento al padre, retaggio delle vecchie realizzazioni; questo puntatore può essere rimosso mantenendo comunque la possibilità di ``risalire'' la struttura dati durante il completamento delle chiamate ricorsive.



\subsection{Mesher}
\label{sec:modules_mesher}
Il \emph{mesher} è il componente che, a partire dai dati forniti dal \emph{miller}, crea la mesh 3D dell'oggetto lavorato. Le interfacce verso il \emph{mesher} e verso il modulo che visualizzerà la scena sono definite: questo permette la scrittura di \emph{mesher} differenti che possono venir scelti all'avvio del software, ad esempio per visualizzare i voxel non cancellati, piuttosto che una ricostruzione del taglio eseguito, attraverso l'algoritmo Marching Cubes. Tale algoritmo viene descritto in sezione \ref{mcalgo}.

La mesh è costruita con gli strumenti messi a disposizione dalla libreria \osg: si tratterà quindi di un albero sui cui rami e sulle cui foglie sono contenute tutte le informazioni necessarie alla visualizzazione del solo oggetto lavorato. Per la struttura dell'albero si è scelto nuovamente un octree non bilanciato in cui le foglie, stavolta, contengono una molteplicità di voxel: quando il numero di oggetti contenuti in una foglia raggiunge un valore di soglia, viene aggiunto un nuovo livello all'albero, ripartendo tra i nuovi figli così creati il volume di competenza e i voxel ivi contenuti.

Il processo di \emph{meshing} si articola in due fasi: una prima fase di aggiornamento dell'albero della scena e una seconda fase in cui vengono calcolate le mesh vere e proprie. Una volta acquisiti i dati dal \emph{miller}, nella prima fase il \emph{mesher} provvede ad aggiornare la struttura eliminando tutti i voxel non più necessari ed inserendo quelli nuovi o modificati, a patto che siano visibili (vengono cioè ignorate tutte le informazioni riguardanti volumi strettamente interni alla superficie dell'oggetto lavorato). \'E importante notare che le informazioni manipolate in questa prima fase sono strettamente legate ai voxel contenuti nell'octree del \emph{miller} e vengono salvate in uno ``spazio utente'' messo a disposizione dagli oggetti di \osg. Ciò permette di ottenere una complessità computazionale $O(1)$ in cancellazione e aggiornamento e tendente a $O(\log_{8}(\text{\# voxel da visualizzare}/\text{max voxel per foglia}))$ per l'inserimento. L'albero così arricchito di informazioni aggiuntive viene quindi ritornato e, al momento della visualizzazione, scatta la seconda fase di \emph{meshing}: per ciascuna foglia modificata nella fase precedente viene ora creata una mesh vera e propria, contenente tutte le informazioni dei voxel di competenza della foglia stessa. Questa scelta permette di ottimizzare l'uso delle risorse grafiche in quanto limita l'estensione dell'albero della scena e diminuisce il numero di mesh da rappresentare, aumentandone la dimensione.

\subsubsection{L'algoritmo Marching Cubes}
\label{mcalgo} Marching Cubes \cite{Lorensen:1987:MCH:37402.37422} è un algoritmo per estrarre una mesh poligonale a partire da un insieme di voxel, ovvero, data la rappresentazione di un volume, ne ricava la superficie, approssimata per mezzo di poligoni. Tali approssimazioni sono date dalla combinazione e rotazione di 15 configurazioni note a priori, definite nell'articolo citato e riportate in figura \ref{fig:mcpolys}.

\begin{figure}[htp]
	\centering
	\includegraphics[width=.85\textwidth]{img/mcpolys}
	\caption{Le 15 configurazioni originali di Marching Cubes.}
	\label{fig:mcpolys}
\end{figure}

Marching Cubes ha a disposizione un array di $256$ ($ = 2^{\mbox{\# di vertici di un voxel}}$) possibili configurazioni, che indicano quali vertici sono interni alla mesh e quali sono esterni. Ad ogni vertice è associato un bit in una determinata posizione di un byte, per poter essere trattato come intero. Per ogni vertice viene controllata la sua posizione rispetto alla superficie, e al termine del procedimento, cioè quando si conosce la posizione di tutti i vertici, si ricava la combinazione di configurazioni che meglio approssimano la superficie.

\paragraph{Sviluppi futuri.}
La figura \ref{fig:meshing_octree} mostra che garantire performance $O(1)$ nella cancellazione e nell'aggiornamento di dati costa molto in termini di spazio occupato, in quanto necessita di una lista doppiamente concatenata e, per ciascun voxel salvato, di una coppia di \emph{shared pointer}.
\begin{figure}[htp]
	\centering
	\includegraphics[width=.85\textwidth]{img/meshing_octree}
	\caption{Octree usato dal mesher per visualizzare l'oggetto lavorato.}
	\label{fig:meshing_octree}
\end{figure}

Un possibile sviluppo futuro consiste nel portare le operazioni di cancellazione e aggiornamento a complessità logaritmica, limitando la prima fase a marcare come ``modificate'' quelle liste di voxel interessate dai cambiamenti; sarà poi la seconda fase che si occuperà di riallineare il contenuto dell'albero con lo stato di fatto. Così facendo le liste possono essere sostituite da array e le coppie di puntatori con singoli \emph{weak pointer}\footnote{I concetti di \emph{shared} e \emph{weak pointer} sono mutuati dalla libreria boost e spiegati nella relativa documentazione, reperibile all'url \url{http://www.boost.org/doc/libs/1_52_0/libs/smart_ptr/smart_ptr.htm}} che puntano al voxel, risparmiando così memoria RAM.


\subsection{Visualizer}

Il visualizzatore, infine, è il componente che mostra l'elaborazione in corso e il risultato finale. Il programma può operare in due modalità: \verb!mesh! e \verb!box! (in più, è anche disponibile la modalità operativa che esclude la rappresentazione grafica). La prima visualizza la mesh come generata dal Mesher usando Marching Cubes, mentre la seconda sfrutta gli oggetti \verb!Box! di OpenSceneGraph ed è meno accurata, permettendo però di bypassare Marching Cubes.

Il visualizzatore mostra inoltre in tempo reale informazioni relative alla lavorazione, come ad esempio il numero di frame al secondo o l'attivazione o meno del getto d'acqua.




\section{Strumenti usati, prerequisiti e istruzioni}
\subsection{Strumenti usati}
Il progetto è stato sviluppato in \cpp. Per lo sviluppo in ambiente Linux abbiamo usato Eclipse su Ubuntu 12.04, mentre per l'ambiente Windows è stato usato Visual Studio 2010. Lo strumento usato per la compilazione è CMake ($\geq 2.6$).

\subsection{Prerequisiti}
Il progetto è stato sviluppato usando le seguenti librerie:
\begin{itemize}[noitemsep]
  \item Boost ($\geq 1.48$): è una libreria che fornisce diverse funzioni per molteplici scopi, come ad esempio gestione dei thread e gestione dei parametri.
  \item Eigen ($\geq 3.1.1$): è una libreria che mette a disposizione funzioni di algebra lineare;
  \item OpenSceneGraph ($\geq 3.0.0$): è un framework che permette di interfacciarsi alle librerie OpenGL in maniera semplificata ed efficiente \cite{osgbeginner}\cite{osgcookbook}.
\end{itemize}

\subsection{Istruzioni}
Il codice è disponibile all'indirizzo \url{http://code.google.com/p/edt-finalproject-nand/}.

Per compilare il progetto bisogna seguire i seguenti passi:
\begin{enumerate}
  \item portarsi nella cartella \verb!/path/del/progetto/!;
  \item lanciare il comando \verb!cmake flags source/CMakeLists.txt!, dove i \verb!flags! di compilazione possono essere:
    \begin{itemize}[noitemsep]
      \item \verb!-G"Visual Studio 10"! per la compilazione in ambiente Windows;
      \item \verb!-G"Unix Makefiles"! per la compilazione in ambiente Linux;
    \end{itemize}
    \begin{itemize}[noitemsep]
      \item \verb!-D CMAKE_BUILD_TYPE=Debug! per compilare in modalità \verb!debug!;
      \item \verb!-D CMAKE_BUILD_TYPE=Release! per compilare in modalità \verb!release!;
    \end{itemize}
  \item lanciare il comando \verb!make! per compilare.
\end{enumerate}

Per eseguire il progetto, lanciare il comando\\ \verb!/path/del/progetto/CNCSimulator opzioni file_positions!\\ dove:
\begin{itemize}
  \item le opzioni possono essere:
    \begin{enumerate}[noitemsep]
      \item \verb!-s x!, dove \verb!x! è la dimensione minima dei voxel. Minore è \verb!x!, maggiori saranno la precisione della simulazione e il tempo impiegato per completare l'esecuzione;
      \item \verb!-v box|mesh|none!, per specificare il tipo di visualizzazione, rappresentando i voxel come cubi (\texttt{box}), approssimando in maniera più precisa il taglio con l'algoritmo MarchingCubes (\texttt{mesh}) o in modalità solo testuale (\texttt{none});
      \item \verb!-p! lancia il simulatore in pausa;
      \item \verb!-f x!, dove \verb!x! indica il rate di apertura del getto d'acqua per la rimozione dei detriti in eccesso;
      \item \verb!-t x!, dove \verb!x! indica la quantità di materiale da rimuovere prima di attivare il getto d'acqua;
      \item \verb!-h!, per visualizzare il menu di help completo.
    \end{enumerate}
  \item \verb!file_positions! è il file contenente i movimenti da riprodurre.
\end{itemize}

In fase di esecuzione, è possibile modificare il comportamento del simulatore interagendo con esso mediante i seguenti comandi, visibili sul terminale premendo \texttt{h}:
\begin{itemize}[noitemsep]
  \item \verb'r'		esegue la simulazione alla massima velocità possibile, aggiornando il visualizzazore quando possibile;
  \item \verb'1'		il miller esegue una mossa e poi si pone in pausa;
  \item \verb'2'		il miller esegue $10$ mosse e poi si pone in pausa;
  \item \verb'3'		il miller esegue $50$ mosse e poi si pone in pausa;
  \item \verb'p'		pausa;
  \item \verb't'		blocca/attiva gli aggiornamenti delle informazioni visualizzate;
  \item \verb'k'		termina il milling;
  \item \verb'h'		mostra il menu di help;
  \item \verb'ESC'	esce dal simulatore.
\end{itemize}


\section{Conclusioni}
Il programma sviluppato mostra come procede passo dopo passo il lavoro di fresatura, permettendo di regolare precisione della simulazione e del rendering e la velocità di visualizzazione.

Con i file di esempio messi a disposizione, le operazioni vengono portate a termine correttamente e con buone prestazioni, nonostante il software non faccia uso di librerie per il calcolo ad alta prestazione quali \cuda.

La lavorazione in modalità testuale è, come ci si può ragionevolmente attendere, più veloce rispetto alle due modalità in cui è impiegata anche la grafica. Per lavorazioni poco impegnative dal punto di vista computazionale la modalità \texttt{box} e la modalità \texttt{mesh} hanno prestazioni comparabili, mentre, via via che l'elaborazione si fa più processor intensive, l'attivazione del modulo Mesher in modalità \texttt{mesh} comporta un significativo incremento sia del tempo di calcolo che della memoria necessaria a contenere l'Octree. Questo è il prezzo da pagare per una visualizzazione più realistica della processo di fresatura, tuttavia, l'uso di Marching Cubes permette di alleggerire il carico della visualizzazione mediante OpenSceneGraph.

Come si può facilmente inferire osservando le tempistiche di esecuzione e osservando quali sono i moduli più time-demanding, i fattori che maggiormente influiscono sul tempo di calcolo sono certamente l'estensione dell'area di lavoro e la profondità dell'Octree: mentre quest'ultima è determinata dalla dimensione dei voxel, la prima è invece fortemente dipendente dalla lavorazione. Se la zona interessata dall'erosione è limitata ad una porzione ristretta di blocco (file \texttt{positions.txt}) i tempi e l'occupazione di memoria sono molto inferiori ad una simulazione in cui la fresa si sposta su tutto il prodotto (file \texttt{positions2.txt}).

A conclusione di questa relazione si vuole evidenziare che, nonostante il lavoro fatto sia stato molto, è comunque possibile intervenire per migliorare ulteriormente il progetto. In primo luogo è possibile incrementare le prestazioni di Miller e Mesher, rendendo il primo parallelo e ``rallentando'' il secondo per consentire l'uso di strutture dati più leggere e performanti; ci si potrebbe concentrare poi sul miglioramento della gestione dell'input multipiattaforma, modificando il formato dei dati di configurazione, e continuare a lavorare su tutti quegli aspetti già delineati nelle trattazioni specifiche dei vari moduli.


% %aggiunge la bibliografia all'indice
 %%niente sezioni anche qui, ovvio....
\clearpage

% \bibliographystyle{plain}
% \bibliography{biblio}
% \addcontentsline{toc}{section}{Bibliografia}

\end{document}
